\documentclass[12pts,a4paper]{report}
\usepackage[utf8]{inputenc}
\usepackage[T1]{fontenc}
\usepackage[french]{babel}
\usepackage[margin=2cm]{geometry}
\usepackage{amsmath, amssymb, amsthm}
\usepackage{enumitem}
\usepackage{hyperref}
\newtheorem{definition}{Définition}
\newtheorem{exemple}{Exemple}
\newtheorem{proposition}{Proposition}
\newtheorem{remarque}{Remarque}
\newtheorem{preuve}{Preuve}
\newtheorem{corollaire}{Corollaire}
\newtheorem{theoreme}{Theoreme}
\title{\LARGE\textbf{Annexes sur les extensions de corps et limites directes/inverses}}
\author{HAJASOA Fanantenana M1 MAFI}
\date{}

\begin{document}
\Large
\maketitle

\section*{Annexe A - Extensions de corps}

\subsection*{1. Corps et sous-corps}

Dans tout ce cours, nous entendrons par corps un anneau commutatif \( K \) vérifiant \( K^* = K \setminus \{0\} \).

\begin{definition}[A.1.1]
Soit \( K \) un corps. Une extension de \( K \) est un couple \((L,j)\) où \( L \) est un corps et \( j : K \to L \) un homomorphisme de corps.
\end{definition}
Comme un homomorphisme de corps est injectif (les seuls idéaux de \( K \) sont l'idéal nul et \( K \)), on peut identifier \( K \) et \( j(K) \) et considérer \( K \) comme un sous-corps de \( L \). Ainsi, dans toute la suite, une extension d'un corps \( K \) est un corps \( L \) tel que \( L \supseteq K \). On écrit aussi \( L/K \) pour dire que \( L \) est une extension de \( K \).

Il est clair que si \( L \) est une extension de \( K \) alors \( L \) est un \( K \)-espace vectoriel et aussi une \( K \)-algèbre.

\begin{definition}[A.1.2]
On appelle degré de l'extension \( L/K \) et on note \([L:K]\) la dimension de \( L \) en tant que \( K \)-espace vectoriel.
\end{definition}

\begin{definition}[A.1.3]
On dit que le corps \( K \) est premier s'il n'admet pas de sous-corps propre.
\end{definition}

\begin{exemple}[A.1.4]
\begin{itemize}
\item L'intersection des sous-corps de \( K \) est un corps premier. C'est le sous-corps premier de \( K \).
\item Le corps \( \mathbb{Q} \) des rationnels est un corps premier. En effet, un sous-corps de \( \mathbb{Q} \) contient le sous-anneau de \( \mathbb{Q} \) engendré par 1. Donc il contient \( \mathbb{Z} \).
\item Pour \( p \) premier, d'après le théorème de Lagrange, le corps \( \mathbb{Z}/p\mathbb{Z} \) est premier.
\end{itemize}
\end{exemple}

\begin{proposition}[A.1.5]
Soit \( K \) un corps.
\begin{enumerate}[label=(\alph*)]
\item Si \( \text{car}(K) = 0 \) alors le sous-corps premier de \( K \) est isomorphe à \( \mathbb{Q} \).
\item Si \( \text{car}(K) = p \) (premier) alors le sous-corps premier de \( K \) est isomorphe à \( \mathbb{Z}/p\mathbb{Z} \).
\end{enumerate}
\end{proposition}

\begin{proof}
Il suffit de considérer l'homomorphisme d'anneaux \( \mathbb{Z} \to K \) tel que \( f(n) = n \cdot 1 \) en se rappelant que la caractéristique d'un anneau intègre est, soit nulle, soit un nombre premier \( p \) (voir [Raz25a]). D'après le premier théorème d'isomorphisme, \( K \) contient, soit un sous-corps isomorphe à \( \mathbb{Z}/p\mathbb{Z} \), soit un sous-anneau isomorphe à \( \mathbb{Z} \). \qedhere
\end{proof}

\begin{definition}[A.1.6]
Soit l'extension \( L/K \). Un \( K \)-automorphisme de \( L \) est un automorphisme de \( L \) en tant que \( K \)-algèbre. C'est donc un automorphisme du corps \( L \) qui laisse invariant les éléments de \( K \). On note \( \text{Gal}(L/K) \) l'ensemble des \( K \)-automorphismes de \( L \). C'est un groupe pour la loi de composition des applications. On l'appelle le groupe de Galois de l'extension \( L/K \).
\end{definition}

\begin{proposition}[A.1.7]
Soient \( K \) un corps et \( P \) son sous-corps premier. On a \( \text{Aut}(K) = \text{Gal}(K/P) \).
\end{proposition}

\begin{proof}
Il est clair que \( \text{Gal}(K/P) \subseteq \text{Aut}(K) \).

Soit \( f \in \text{Aut}(K) \). On a \( K^f = \{x \in K, f(x) = x\} \) est un sous-corps de \( K \). Ainsi \( P \subseteq K^f \) et \( f \in \text{Gal}(K/P) \). \qedhere
\end{proof}

\begin{exemple}[A.1.8]
On a \( \text{Aut}(\mathbb{R}) = \text{Gal}(\mathbb{R}/\mathbb{Q}) = \{\text{id}_{\mathbb{R}}\} \).

En effet, d'après la proposition précédente, on a déjà \( \text{Aut}(\mathbb{R}) = \text{Gal}(\mathbb{R}/\mathbb{Q}) \). Soit \( x \in \mathbb{R} \) avec \( x > 0 \). Si \( f \in \text{Gal}(\mathbb{R}/\mathbb{Q}) \) alors \( f(x) = f(\sqrt{x^2}) = f(\sqrt{x})^2 \geq 0 \). Comme \( f \) est injectif, on a \( f(x) > 0 \). Il vient que si \( a - b > 0 \) alors \( f(a - b) = f(a) - f(b) > 0 \) et \( f \) est strictement croissant.

Soit alors \( x \in \mathbb{R} \setminus \mathbb{Q} \). Si \( f(x) < x \) alors il existe \( r \in \mathbb{Q} \) tel que \( f(x) < r < x \). Dans ce cas, \( f(f(x)) < f(r) = r < f(x) \). Ce qui est absurde. De la même manière, on ne peut pas avoir \( f(x) > x \).
\end{exemple}

\begin{definition}[A.1.9]
Soit l'extension \( L/K \) et soit \( A \subseteq L \). On désigne par \( K(A) \) le sous-corps de \( L \) engendré par \( K \) et \( A \). On dit aussi que \( K(A) \) est le sous-corps de \( L \) engendré par \( A \) sur \( K \). C'est l'intersection des sous-corps de \( L \) contenant \( K \) et \( A \).

Si \( A = \{\alpha_1, ..., \alpha_r\} \), on écrit simplement \( K(\alpha_1, ..., \alpha_r) \) et on dit que \( K(\alpha_1, ..., \alpha_r)/K \) est une extension de type fini. Si \( A = \{\alpha\} \), on dit que \( K(\alpha) \) est une extension simple de \( K \).
\end{definition}

\subsection*{2. Éléments algébriques, éléments transcendants}

Soit une extension \( K \subseteq L \) et soit \( \alpha \in L \). Le plus petit sous-anneau de \( L \) contenant \( K \) et \( \alpha \) est l'ensemble \( K[\alpha] \) des expressions polynomiales en \( \alpha \). Son corps des fractions, noté \( K(\alpha) \), est le plus petit sous-corps de \( L \) contenant \( K \) et \( \alpha \). Soit \( X \) une indéterminée et soit l'homomorphisme d'anneaux

\[
\varepsilon_\alpha : K[X] \longrightarrow K[\alpha]
\]

défini par \( \varepsilon_\alpha (a) = a \) si \( a \in K \) et \( \varepsilon_\alpha (X) = \alpha \). Il est clair que \( \varepsilon_\alpha \) est surjectif. Deux cas peuvent se présenter selon que \( \ker (\varepsilon_\alpha) \) est nul ou non.

\begin{definition}[A.2.1]
On dit que \( \alpha \) est transcendant sur \( K \) si \( \ker (\varepsilon_\alpha) = (0) \), et \( \alpha \) est algébrique sur \( K \) si \( \ker (\varepsilon_\alpha) \neq (0) \).
\end{definition}

\begin{enumerate}[label=(\alph*)]
\item Si \( \alpha \) est transcendant sur \( K \), il n'existe pas de polynôme non nul de \( K[X] \) qui admet \( \alpha \) comme racine. Dans ce cas, \( \varepsilon_\alpha \) est un isomorphisme d'anneaux principaux \( K[X] \longrightarrow K[\alpha] \).

\item Si \( \alpha \) est algébrique sur \( K \), il existe un polynôme non nul de \( K[X] \) qui s'annule en \( \alpha \). Soit \( P \) un générateur de \( \ker (\varepsilon_\alpha) \) dans l'anneau principal \( K[X] \). Puisque \( K[X] / (P) \) est isomorphe à l'anneau intègre \( K[\alpha] \), l'idéal \( (P) \) est un idéal premier, donc maximal, de \( K[X] \) et \( P \) est un élément irréductible de \( K[X] \). Il s'ensuit que \( K[\alpha] \) est un corps et \( K[\alpha] = K(\alpha) \).
\end{enumerate}

\begin{definition}[A.2.2]
Si \( \alpha \) est algébrique sur \( K \), on appelle polynôme minimal de \( \alpha \) sur \( K \) et on note \( m_{\alpha, K} \), ou simplement \( m_\alpha \) si le contexte est clair, le générateur unitaire de \( \ker (\varepsilon_\alpha) \). Le degré de \( \alpha \) est le degré de son polynôme minimal.
\end{definition}

\begin{proposition}[A.2.3]
Soit l'extension \( L/K \) et soit \( \alpha \in L \). Les assertions suivantes sont équivalentes :
\begin{enumerate}[label=(\alph*)]
\item L'élément \( \alpha \) est algébrique sur \( K \).
\item On a l'égalité \( K[\alpha] = K(\alpha) \).
\item Le \( K \)-espace vectoriel \( K(\alpha) \) est de dimension finie.
\end{enumerate}
Plus précisément, si \( [K(\alpha) : K] \) est fini alors \( [K(\alpha) : K] = \deg \alpha \).
\end{proposition}

\begin{proof}
On a déjà (1)implique(2). Réciproquement, si \( K[\alpha] = K(\alpha) \) alors l'homomorphisme \( \varepsilon_\alpha \) n'est pas injectif et \( \alpha \) est algébrique sur \( K \).

(1)implique(3) Soit \( x = F(\alpha) \) avec \( F \in K[X] \). On a \( F(X) = Q(X)m_\alpha(X) + R(X) \) avec \( \deg R < \deg m_\alpha \) ou \( R = 0 \). Ainsi \( x = R(\alpha) \) et \( (1, \alpha, ..., \alpha^{\deg \alpha - 1}) \) est un système générateur de \( K(\alpha) \).

(3)implique (1) Si \( [K(\alpha) : K] = n \), fini, la famille \( (1, \alpha, ..., \alpha^n) \) n'est pas libre sur \( K \). A partir d'une relation de dépendance linéaire entre les éléments de cette famille on obtient un polynôme non nul de \( K[X] \), de degré \( n \), qui s'annule en \( \alpha \).

Enfin, si \( [K(\alpha) : K] \) est fini, le système générateur \( (1, \alpha, ..., \alpha^{\deg \alpha - 1}) \) est un système libre sinon il existerait un polynôme non nul \( F \in K[X] \) tel que \( \deg F < \deg m_\alpha \) et \( F \in (m_\alpha) = \ker (\varepsilon_\alpha) \). \qedhere
\end{proof}

\subsection*{3. Extensions finies}

\begin{definition}[A.3.1]
L'extension \( L/K \) est dite finie si la dimension de \( L \) en tant que \( K \)-espace vectoriel est finie.
\end{definition}

\begin{remarque}[A.3.2]
Une extension de type fini n'est pas nécessairement finie. Si \( L = K(X) \), le corps des fractions rationnelles en l'indéterminée \( X \) alors \( X \) n'est pas algébrique sur \( K \) de sorte que la dimension de \( L \) en tant que \( K \)-espace vectoriel n'est pas finie.
\end{remarque}

\begin{definition}[A.3.3]
On appelle corps de nombres toute extension finie du corps des rationnels \( \mathbb{Q} \).
\end{definition}

\begin{proposition}[A.3.4]
Si \( L/K \) et \( K/H \) sont des extensions finies alors \( L/H \) est une extension finie et \([L:H] = [L:K][K:H]\).
\end{proposition}

\begin{proof}
Soient \( (k_i)_{1\leq i\leq m} \) une \( H \)-base de \( K \) et \( (l_j)_{1\leq j\leq n} \) une \( K \)-base de \( L \). Si \( \gamma \in L \) alors

\[
\gamma = \sum_{j=1}^n \alpha_j l_j \quad \text{avec } \alpha_j \in K,
\]
\[
\alpha_j = \sum_{i=1}^m \alpha_{ij} k_i \quad \text{avec } \alpha_{ij} \in H.
\]

Et, en regroupant

\[
\gamma = \sum_{j=1}^n \sum_{i=1}^m \alpha_{ij} k_i l_j.
\]

Ainsi, \( (k_i l_j)_{1\leq i\leq m, 1\leq j\leq n} \) est un système générateur de \( L \) sur \( H \).

Si \( \sum_{i,j} \alpha_{ij} k_i l_j = \sum_j (\sum_i \alpha_{ij} k_i) l_j = 0 \) alors, comme \( \sum_i \alpha_{ij} k_i \in K \), pour tout \( j \) on a \( \sum_i \alpha_{ij} k_i = 0 \). Donc \( \alpha_{ij} = 0 \) pour tout \( i, j \) et \( (k_i l_j)_{1\leq i\leq m, 1\leq j\leq n} \) est une \( H \)-base de \( L \). \qedhere
\end{proof}

\begin{corollaire}[A.3.5]
Si \( [L:K] = n \), fini, alors tout élément de \( L \) est algébrique sur \( K \) et de degré un diviseur de \( n \).
\end{corollaire}

\begin{proof}
Si \( x \in L \), alors la famille \( (1,x,...,x^n) \) est \( K \)-liée. On obtient alors un élément non nul de \( K[X] \) qui s'annule en \( x \). Il suffit alors de remarquer que \( n = [L:K] = [L:K(x)][K(x):K] \).

Soit l'extension finie \( L/K \). Si \( \alpha \in L \), on considère le \( K \)-endomorphisme \( \delta_\alpha \) de \( L \) défini par \( \delta_\alpha (x) = \alpha x \). \qedhere
\end{proof}

\begin{definition}[A.3.6]
On appelle trace (resp. norme) de \( \alpha \) relativement à \( L/K \) et on note \( \text{Tr}_{L/K}(\alpha) \) (resp. \( N_{L/K}(\alpha) \)) la trace (resp. le déterminant) de \( \delta_\alpha \).

\[
\text{Tr}_{L/K}(\alpha) = \text{Tr}(\delta_\alpha) \quad N_{L/K}(\alpha) = \det(\delta_\alpha).
\]
\end{definition}

\begin{definition}[A.3.7]
On appelle polynôme caractéristique de \( \alpha \) relativement à \( L/K \) et on note \( c_{\alpha, L/K} \) (ou \( c_\alpha \) quand le contexte est clair) le polynôme caractéristique de \( \delta_\alpha \),

\[
c_{\alpha, L/K}(X) = \det(X\text{id} - \delta_\alpha).
\]
\end{definition}

\begin{proposition}[A.3.8]
On a

\[
c_{\alpha, L/K}(X) = m_{\alpha, K}(X)^{[L:K(\alpha)]}.
\]
\end{proposition}

\begin{proof}
Supposons d'abord que \( L = K(\alpha) \). D'après le théorème de Cayley-Hamilton, \( c_\alpha (\delta_\alpha) = 0 \). On a alors \( c_\alpha (\alpha) = 0 \) et \( m_\alpha \mid c_\alpha \). Comme ces polynômes sont unitaires et de même degré, \( c_\alpha = m_\alpha \).

Dans le cas général, soit \( (k_i) \) une \( K \)-base de \( K(\alpha) \) et soit \( (l_j) \) une \( K(\alpha) \)-base de \( L \). Comme dans la preuve du théorème A.3.4, la famille \( (k_i l_j) \) est une \( K \)-base de \( L \). On a \( \delta_\alpha (k_i) = \sum_j a_{ij} k_j \) et d'après la première partie, le polynôme caractéristique de la matrice \( A = (a_{ij}) \), qui est une matrice carrée \( [K(\alpha):K] \times [K(\alpha):K] \), n'est autre que \( m_\alpha \). Maintenant, \( \delta_\alpha (k_i l_j) = \alpha k_i l_j = \sum_j a_{ij} k_j l_j \), de sorte que la matrice de \( \delta_\alpha \) dans la base \( (k_i l_j) \) est formée de \( [L:K(\alpha)] \) blocs diagonaux tous égaux à \( A \). D'où le résultat. \qedhere
\end{proof}

\subsection*{4. Extensions algébriques}

\begin{definition}[A.4.1]
L'extension \( L/K \) est dite algébrique si tout élément de \( L \) est algébrique sur \( K \). Dans le cas contraire, on dit que l'extension \( L/K \) est transcendante.
\end{definition}

On a déjà vu que si l'extension \( L/K \) est finie alors elle est algébrique (voir corollaire A.3.5).

\begin{proposition}[A.4.2]
Soit \( L = K(\alpha_1, \ldots, \alpha_n) \) une extension de type fini de \( K \). Si les \( \alpha_i \) sont algébriques sur \( K \) alors \( L/K \) est finie (donc algébrique) et \( L = K[\alpha_1, \ldots, \alpha_n] \).
\end{proposition}

\begin{proof}
On raisonne par récurrence sur \( n \). On a déjà vu le résultat pour \( n = 1 \). Supposons alors \( n > 1 \) et posons \( H = K(\alpha_1, \ldots, \alpha_{n-1}) \). Par hypothèse de récurrence, \( H/K \) est fini et \( H = K[\alpha_1, \ldots, \alpha_{n-1}] \). Puisque \( \alpha_n \) est algébrique sur \( H \), on a \( H(\alpha_n)/H \) fini et \( L = H(\alpha_n) = H[\alpha_n] \). Il vient que \( [L : K] = [L : H][H : K] \) est fini. \qedhere
\end{proof}

\begin{proposition}[A.4.3]
Si \( L/K \) et \( K/H \) sont algébriques alors \( L/H \) est algébrique.
\end{proposition}

\begin{proof}
Soit \( \alpha \in L \) et soit \( m_{\alpha,K}(X) = X^n + a_{n-1}X^{n-1} + \cdots + a_1X + a_0 \) son polynôme minimal sur \( K \). Considérons alors le corps \( M = H(a_0, \ldots, a_{n-1}) \). Puisque les \( a_i \) sont algébriques sur \( H \) (ils appartiennent à \( K \)), l'extension \( M/H \) est finie. Comme \( \alpha \) est évidemment algébrique sur \( M \), l'extension \( M(\alpha)/M \) est finie et \( [M(\alpha) : H] = [M(\alpha) : M][M : H] \) est fini. Ce qui montre que \( \alpha \) est algébrique sur \( H \). \qedhere
\end{proof}

En examinant les résultats précédents, une question se pose. Soit \( K(\alpha_1, \ldots, \alpha_n) \) une extension de type fini de \( K \). Si \( K(\alpha_1, \ldots, \alpha_n) = K[\alpha_1, \ldots, \alpha_n] \), peut-on dire que les \( \alpha_i \) sont algébriques sur \( K \) ?

\begin{remarque}[A.4.4]
Nous dirons que l'extension \( L \) de \( K \) est une \( K \)-algèbre de type fini s'il existe \( \alpha_1, \ldots, \alpha_n \in L \) tels que \( L = K[\alpha_1, \ldots, \alpha_n] \).

Si l'extension \( L/K \) est de type fini, \( L \) n'est pas nécessairement une \( K \)-algèbre de type fini. C'est le cas par exemple du corps des fractions rationnelles en l'indéterminée \( X \). En effet, si \( K(X) = K[\tau_1, \ldots, \tau_n] \) et si \( D \) est un dénominateur commun des \( \tau_j \) alors pour tout \( z \in K(X) \) il existe \( N \in \mathbb{N} \) tel que \( D^N z \in K[X] \). Ce qui est évidemment impossible en prenant \( z = 1/c \) avec \( c = 1 + d_1d_2 \cdots d_t \) où les \( d_i \) sont les diviseurs irréductibles de \( D \) dans l'anneau factoriel \( K[X] \).
\end{remarque}

\begin{theoreme}[A.4.5 (Zariski)]
Soit l'extension \( L/K \). Si \( L \) est une \( K \)-algèbre de type fini alors l'extension \( L/K \) est finie (donc algébrique).
\end{theoreme}

\begin{proof}
Supposons que \( L = K[\alpha_1, \ldots, \alpha_n] \) et raisonnons par récurrence sur \( n \). Le cas \( n = 1 \) est résolu par la proposition A.2.3. Supposons alors \( n > 1 \) et le résultat vrai pour toute extension engendrée par \( t \leq n - 1 \) éléments en tant qu'algèbre sur un corps quelconque. Posons \( K_1 = K(\alpha_1) \). Par hypothèse de récurrence, \( L = K_1[\alpha_2, \ldots, \alpha_n] \) est algébrique sur \( K_1 \). Si \( \alpha_1 \) est algébrique sur \( K \) alors on a le résultat. Supposons alors \( \alpha_1 \) transcendant sur \( K \).

Pour \( i \geq 2 \), on a une équation
\[
\alpha_i^m + a_{i1}\alpha_i^{m-1} + \cdots + a_{im-1} = 0 \quad \text{avec } a_{ij} \in K_1.
\]

Si \( a \) est un dénominateur commun des \( a_{ij} \), on a
\[
(a\alpha_i)^m + a_{i1}a(a\alpha_i)^{m-1} + \cdots + a^{m}a_{im-1} = 0,
\]
de sorte que les \( a\alpha_i \) sont entiers sur \( K[\alpha_1] \). Il vient que pour tout \( z \in L \), il existe \( N \in \mathbb{N} \) tel que \( a^N z \) soit entier sur \( K[\alpha_1] \). Comme \( K[\alpha_1] \) est intégralement clos (car factoriel), pour tout \( z \in L = K[\alpha_1, \ldots, \alpha_n] \), il existe \( N \in \mathbb{N} \) tel que \( a^N z \in K[\alpha_1] \). En particulier ce résultat serait vrai pour \( z \in K(\alpha_1) \). Ce qui est impossible car \( K(\alpha_1) \) est isomorphe au corps des fractions rationnelles \( K(X) \) et il suffit de prendre \( z = 1/c \) avec \( c \) premier avec \( a \) comme dans la remarque A.4.4. \qedhere
\end{proof}

\begin{proposition}[A.4.6]
Soit l'extension \( L/K \) et soit \( F \) l'ensemble des éléments de \( L \) qui sont algébriques sur \( K \). L'ensemble \( F \) est un sous-corps de \( L \) qui contient \( K \). C'est la fermeture algébrique de \( K \) dans \( L \).
\end{proposition}

\begin{proof}
Il est clair que \( K \subseteq F \subseteq L \). Soient \( \alpha, \beta \in F \). D'après la proposition A.4.2, \( K(\alpha, \beta) \) est une extension algébrique de \( K \) et \( \alpha \pm \beta \), \( \alpha \beta, \alpha / \beta \) (si \( \beta \neq 0 \)) sont dans \( K(\alpha, \beta) \). \qedhere
\end{proof}

\subsection*{5. Extensions transcendantes}

\begin{definition}[A.5.1]
Soit l'extension \( L/K \). Les éléments \( x_1, \ldots, x_n \) de \( L \) sont algébriquement indépendants sur \( K \) s'il n'existe pas de polynôme non nul \( f \in K[X_1, \ldots, X_n] \) tel que \( f(x_1, \ldots, x_n) = 0 \). Autrement dit, l'anneau engendré par \( K \) et les \( x_i \) est isomorphe à \( K[X_1, \ldots, X_n] \).

Dire que \( x \) est algébriquement libre sur \( K \) signifie que \( x \) est transcendant sur \( K \).
\end{definition}

Il est clair que si la famille des \( x_1, \ldots, x_n \) est algébriquement indépendante sur \( K \) alors il en est de même de toute partie \( \{x_{i_1}, \ldots, x_{i_s}\} \).

\begin{proposition}[A.5.2]
Soit l'extension \( L/K \) et soient \( x_1, \ldots, x_n \) des éléments deux à deux distincts de \( L \). Soit \( s \) tel que \( 1 < s < n \). Alors \( x_1, \ldots, x_n \) sont algébriquement indépendants sur \( K \) si et seulement si \( x_1, \ldots, x_s \) sont algébriquement indépendants sur \( K \) et \( x_{s+1}, \ldots, x_n \) sont algébriquement indépendants sur \( K(x_1, \ldots, x_s) \).
\end{proposition}

\begin{proof}
1) Supposons \( x_1, \ldots, x_n \) algébriquement indépendants sur \( K \). Il en est de même de \( x_1, \ldots, x_s \). S'il existe

\[
f \in K[x_1, \ldots, x_s][Y_{s+1}, \ldots, Y_n], \quad f \neq 0
\]

tel que \( f(x_{s+1}, \ldots, x_n) = 0 \). Il existe \( h \in K[x_1, \ldots, X_s] \) tel que \( h(x_1, \ldots, x_s) \) soit un dénominateur commun des coefficients de \( f \). Soit \( f_1 \in K[x_1, \ldots, X_s, Y_{s+1}, \ldots, Y_n] \) tel que \( f_1(x_1, \ldots, x_s, Y_{s+1}, \ldots, Y_n) = f \). On a \( 0 \neq hf_1 = g \in K[x_1, \ldots, X_s, Y_{s+1}, \ldots, Y_n] \) tel que \( g(x_1, \ldots, x_s, x_{s+1}, \ldots, x_n) = 0 \), contrairement à notre hypothèse.

2) Supposons que \( x_1, \ldots, x_s \) sont algébriquement indépendants sur \( K \) et que \( x_{s+1}, \ldots, x_n \) sont algébriquement indépendants sur \( K(x_1, \ldots, x_s) \). S'il existe \( f \in K[x_1, \ldots, X_n] \) tel que \( f(x_1, \ldots, x_n) = 0 \) alors, si

\[
g = f(x_1, \ldots, x_s, X_{s+1}, \ldots, X_n) \in K[x_1, \ldots, x_s][X_{s+1}, \ldots, X_n]
\]

n'est pas nul, \( x_{s+1}, \ldots, x_n \) ne seraient algébriquement indépendants sur \( K(x_1, \ldots, x_s) \) car \( g(x_{s+1}, \ldots, x_n) = 0 \).

Donc \( g = f(x_1, \ldots, x_s, X_{s+1}, \ldots, X_n) = 0 \) et les coefficients de \( g \) sont tous nuls. Or les coefficients de \( g \) sont les valeurs prises en \( X_1 = x_1, \ldots, X_s = x_s \) de polynômes \( T_i \in K[x_1, \ldots, X_s] \). Comme \( x_1, \ldots, x_s \) sont algébriquement indépendants sur \( K \), les polynômes \( T_i \) sont tous nuls. Mais, ces polynômes \( T_i \) sont les coefficients de \( f \) en considérant \( f \) comme élément de \( K[x_1, \ldots, X_s][X_{s+1}, \ldots, X_n] \). Il vient que \( f = 0 \). \qedhere
\end{proof}

\begin{definition}[A.5.3]
Soit l'extension \( L/K \). Une famille \( (x_1, \ldots, x_n) \) d'éléments de \( L \) est appelée base de transcendance de \( L \) sur \( K \) si
\begin{enumerate}[label=(\alph*)]
\item les éléments \( x_1, \ldots, x_n \) sont algébriquement indépendants sur \( K \),
\item le corps \( L \) est algébrique sur \( K(x_1, \ldots, x_n) \).
\end{enumerate}

Une base de transcendance est une base de transcendance pure si elle engendre l'extension.
\end{definition}

\begin{exemple}[A.5.4]
La famille \( (X_1, \ldots, X_n) \) est une base de transcendance pure de \( K(X_1, \ldots, X_n) \) sur \( K \). Par contre \( \{X^2\} \) est une base de transcendance de \( K(X) \) sur \( K \) (\( X \) est racine de \( T^2 - X^2 \)) mais ce n'est pas une base de transcendance pure.
\end{exemple}

\begin{remarque}[A.5.5]
Une famille \( B \) d'éléments de \( L \) est une base de transcendance de \( L \) sur \( K \), si et seulement si \( B \) est une famille algébriquement libre maximale.
\end{remarque}

\begin{proposition}[A.5.6]
Soit \( L/K \) une extension de type fini. Deux bases de transcendance de \( L \) sur \( K \) ont le même nombre d'éléments.
\end{proposition}

\begin{proof}
Si \( L = K(S) \) avec \( S \) une partie finie de \( L \) alors, une partie maximale de \( S \) formée d'éléments algébriquement indépendants sur \( K \) est une base de transcendance de \( L \) sur \( K \). Ainsi \( L \) admet une base de transcendance sur \( K \) formée d'un nombre fini \( n \) d'éléments.

On raisonne par récurrence sur \( n \) en montrant que : pour toute extension \( H/F \) ayant une base de transcendance de cardinal \( n \), alors toute partie de \( H \) formée d'éléments algébriquement indépendants sur \( F \) est de cardinal \( \leq n \).

Si \( n = 0 \) alors \( L \) est algébrique sur \( K \) et on a le résultat.

Supposons alors \( n \geq 1 \). Soient \( (x_1, \ldots, x_n) \) une base de transcendance de \( H \) sur \( F \) et \( y_1, \ldots, y_m \) des éléments de \( H \) algébriquement indépendants sur \( F \). On complète \( \{y_1\} \) par des \( x_i \) pour avoir une partie maximale \( \{y_1, x_{i_1}, \ldots, x_{i_s}\} \) formée d'éléments algébriquement indépendants sur \( F \) (donc une base de transcendance de \( H \) sur \( F \)). Par maximalité de la famille \( (x_i) \), on a \( s \leq n - 1 \). Soit le corps \( F(y_1) \). La famille \( (x_{i_1}, \ldots, x_{i_s}) \) est une base de transcendance de \( H \) sur \( F(y_1) \). Par hypothèse de récurrence, \( m - 1 \leq s \leq n - 1 \). Il s'ensuit que \( m \leq n \). \qedhere
\end{proof}

\begin{definition}[A.5.7]
Le cardinal d'une base de transcendance d'une extension de type fini \( L/K \) s'appelle le degré de transcendance de \( L \) sur \( K \). On le note degtr \( L/K \). On a degtr \( L/K = 0 \) si et seulement si \( L \) est algébrique sur \( K \).
\end{definition}
     \begin{center}
     	MERCI !
     \end{center}
\end{document}

